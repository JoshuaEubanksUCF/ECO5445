\documentclass[11pt]{paper}
\usepackage{geometry}
\usepackage{hyperref}
\usepackage{xcolor}
\geometry{
  top = 1in
  , bottom = 1in
  , left = 1in
  , right = 1in
  }
\hypersetup{
	colorlinks=true,
	linkcolor=blue,
	filecolor=magenta,
	urlcolor=cyan,
}

\begin{document}
\title{ECO 5445 Section B001: Introduction to Business Analytics}
\author{University of Central Florida --- Department of Economics}

\maketitle
\hrulefill

\begin{flushleft}
{\color{red} \textbf{In light of the ever-changing environment, the syllabus may be revised several times throughout the semester.}}
\end{flushleft}

\section*{Course Information}
\begin{flushleft}
\begin{tabular}{| l | l |}\hline
 Course & ECO 5445-22Summer B001 \\\hline
 Term & Summer 2022 \\\hline
 Meeting Time & MW 6:00pm-8:50pm\\\hline
 Location & BA1 207\\\hline
 Credit Hours & 3 \\\hline
\end{tabular}
\end{flushleft}

\section*{Instructor Information}
\begin{flushleft}
\begin{tabular}{| l | l |}\hline
 Instructor & Joshua L. Eubanks \\\hline
 Office & \href{https://ucf.zoom.us/j/2438809427}{Virtual Office} and BA2 302G \\\hline
 Hours & \href{https://calendly.com/ucf-office-hours-jeubanks}{By appointment} \\\hline
 Email & Email through Webcourses \\\hline
\end{tabular}
\end{flushleft}


\section*{Course Materials}
 

\subsection*{Textbooks}
\textit{Practical Programming: An Introduction to Computer Science Using Python 3.6, 3rd Edition} by Paul Gries, Jennifer Campbell, Jason Montojo. The Pragmatic Bookshelf: Raleigh, NC, 2017 (ISBN-13: 978-1680502688, ISBN-10: 1680502689)\\

\textit{R in Action: Data Analysis and Graphics with R, 1st Edition} 1st Edition by Robert Kabacoff. Manning Publications Company, Shelter Island, NY, 2015 (ISBN-13: 978-1935182399, ISBN-10: 1935182390)\\

\textit{Business Data Science: Combining Machine Learning and Economics to Optimize, Automate, and Accelerate Business Decisions, 1st Edition} by Matt Taddy. McGraw-Hill: New York, NY, 2019 (ISBN-13: 978-1260452778 ISBN-10: 1260452778)\\

  
\section*{Course Description}
This course is designed to introduce you to the main tools of business analytics. It is aimed at students interested in pursuing a career in business analytics, with an emphasis on accounting and financial data. The course begins with fundamental concepts in computing, to gain an understanding of the operations and functions that you can use to manipulate a variety of data types. You will learn how to design computer programs that obtain data by web-scraping, then read data, manipulate data, perform calculations with data and apply these to problems in business analytics. The course progresses through regression analysis, logistic regression, and other classification models, and concludes with advanced techniques, including dimension-reduction and nonparametric techniques for text analytics. \\

Since this is a 6 week course, you will need to dedicate more time to this than a standard class. The UCF standard is 2x the number of classroom hours in a week. Since we meet for 6 hours, you are expected to study at least 12hr/week.


\section*{Student Learning Outcomes}
By the end of this course, you should be able to:
\begin{itemize}
	\item	understand the types of data in Python and R and the corresponding operations;
  \item use Python modules and R packages to solve problems;
	\item write new functions in Python and R;
	\item obtain, prepare, and analyze data to solve a business problem; and
	\item build models to explain and predict business phenomena.
  
\end{itemize}

\section*{Enrollment Requirements}
The formal requirement is admission to the Master of Science in Accounting program, which normally includes undergraduate coursework equivalent to UCF courses QMB 3003 and QMB 3200, Quantitative Business Tools I and II. The true requirement, however, is a quantitative mindset. Computer programming is a quantitative exercise and is suitable for those with an interest in quantitative problem solving.

\section*{Course Structure}
I will update the home page each Monday morning containing the content that must be completed before the following week.\\

The tentative schedule is as follows:   

\begin{center}
\begin{tabular}{| l | l | l |}\hline
 Day & Material & Chapter(s) \\\hline 
 Jun 27 & Introduction, Using and Designing functions & Ch 1 - Ch 3 (PP) \\
 Jun 29 & Text and strings, boolean variables, lists, loops, and modules & Ch 4-6, 8, and 9 (PP) \\
 Jul 04 & No Class - Independence Day & --- \\
 Jul 06 & Reading/Writing Files and Web scraping & Ch 10 (PP), Beautiful Soup Doc.\\
 Jul 11 & Getting StaRted with R and Data management & Ch 2-5 (R) \\
 Jul 13 & Basic Graphs and Statistics & Ch 6-7 (R); Ch 1 (BDS)\\
 Jul 18 & Regression and Logistic Regression & Ch 8, Ch 13 (R); Ch 2 (BDS)\\
 Jul 20 & Simulation and Regularization & Ch 12 (R); Ch 3 (BDS)\\
 Jul 25 & Classification and Dimension Reduction & Ch 4, Ch 7 (BDS)\\
 Jul 27 & Text as Data and Non-Parametric Models & Ch 8-9 (BDS)\\
 Aug 01 & TBA & ---\\
 Aug 03 & TBA & ---\\\hline
\end{tabular}
\end{center}

% There will be 3 assignments that I will require a handwritten submission. The FASFA attendance assignment, the midterm exam, and the final exam. There are two options:

% \begin{itemize}
% 	\item If you have a printer and a scanner:
% 		\begin{itemize}
% 			\item Print each page, complete, then scan to a pdf.
% 		\end{itemize}
% 	\item If you do not have a printer or a scanner:
% 		\begin{itemize}
% 			\item Download OneDrive. All students have access to OneDrive.
% 			\item Even if you do not have an answer, write each question's point value and equations (if applicable) on a sheet of paper.
% 			\item Using proper angles and lighting, use CamScanner to take pictures of each page. For additional assistance there is a video within the first week of this course.

% 		\end{itemize} 
% \end{itemize}

% These are the only acceptable submission forms, if it is not in order, not rotated properly, unreadable, etc. I will remove points at my discretion. 

\subsection*{Assessment and Grading Procedure}

\begin{flushleft}
\begin{tabular}{ l  l }\hline
 Activity & Percent of Grade \\\hline 
 Assignments &  40\% \\
 Project Stage 1 & 30\% \\
 Project Stage 2 & 30\% \\\hline
\end{tabular}
\end{flushleft}

\subsubsection*{Scale\footnote{I reserve the right to re-scale in a rank-preserving fashion. Grades will be rounded to the nearest whole number.}}
\begin{flushleft}
\begin{tabular}{ l  l }\hline
 Percent & Grade \\\hline 
 100-93 &  A \\
 92.9-90 & A-\\
 89.9-87 & B+ \\
 86.9-83 & B \\
 82.9-80 & B- \\
 79.9-77 & C+ \\
 76.9-73 & C \\
 72.9-70 & C- \\
 69.9-67 & D+ \\
 66.9-63 & D \\
 62.9-60 & D- \\
 $\leq$59.9 & F \\\hline
\end{tabular}
\end{flushleft}

\subsubsection*{Reporting}
Grades will be added via Webcourses to follow student data classification and security standards.

\section*{Policy Statements}
\subsubsection*{Academic Integrity}
The Center for Academic Integrity (CAI) defines academic integrity as a commitment, even in the face of adversity, to five fundamental values: honesty, trust, fairness, respect, and responsibility. From these values flow principles of behavior that enable academic communities to translate ideals into action.\footnote{\url{https://academicintegrity.org/}}\\

The \href{https://osrr.sdes.ucf.edu/}{Office of Student Rights and Responsibilities}\footnote{Located in Ferrell Commons, Room 227}will be notified of any instance of academic misconduct that has occurred inside or outside of the classroom. Students are encouraged to read the \href{https://goldenrule.sdes.ucf.edu/}{Golden Rule Student Handbook}.\\

Students should familiarize themselves with UCF’s Rules of Conduct. According to Section 1, ``Academic Misconduct,'' students are prohibited from engaging in
\begin{enumerate}
\item Unauthorized assistance: Using or attempting to use unauthorized materials, information or study aids in any academic exercise unless specifically authorized by the instructor of record. The unauthorized possession of examination or course-related material also constitutes cheating.
\item Communication to another through written,visual,electronic, or oral means:The presentation of material which has not been studied or learned, but rather was obtained through someone else’s efforts and used as part of an examination, course assignment, or project.
\item Commercial Use of Academic Material: Selling of course material to another person, student, and/or uploading course material to a third-party vendor without authorization or without the express written permission of the university and the instructor. Course materials include but are not limited to class notes, Instructor’s PowerPoints, course syllabi, tests, quizzes, labs, instruction sheets, homework, study guides, handouts, etc.
\item Falsifying or misrepresenting the student’s own academic work.
\item Plagiarism: Using or appropriating another’s work without any indication of the source, thereby attempting to convey the impression that such work is the student’s own.
\item Multiple Submissions: Submitting the same academic work for credit more than once without the express written permission of the instructor.
\item Helping another violate academic behavior standards.
\end{enumerate}
For more information about plagiarism and misuse of sources, see ``Defining and Avoiding Plagiarism: The WPA Statement on Best Practices.''\\


UCF faculty members strive to provide a quality education, and so seek to prevent unethical behavior and when necessary, respond to infringements of academic integrity. Penalties can include a failing grade in an assignment or course, suspension/expulsion from the university, and/or a ``Z'' designation\footnote{More information on Z designation \href{https://goldenrule.sdes.ucf.edu/zgrade}{here}.} on a students official transcript designating academic dishonesty. 

\subsubsection*{Active Duty Military}
Students under active duty in the military will be accommodated as much as possible. Please see me  prior to scheduled military obligations if this applies to you.
\subsubsection*{Attendance/Late Policies}
Late work will not be accepted. Attendance will not be recorded, however, the quizzes administered will not be announced and lecture notes will not be uploaded. As stated in the assignments section, if you show up over 20 minutes after we have started the midterm or final exam, you will \textbf{NOT} be able to take the exam.
\subsubsection*{Emergency Procedure and Campus Safety}
Be aware of your surroundings and be familiar with the necessary actions to take in the event of an emergency. In case of an emergency, dial 911 for assistance. All classrooms contain an emergency procedure guide and is available \href{http://emergency.ucf.edu/emergency_guide.html}{online}. I advise signing up for text alerts from UCF if not already registered. Steps are below:
\begin{itemize}
	\item Log in to myUCF
	\item Click the `Student Self Service' tab
	\item Click the `Personal Information' tab
	\item Click the `UCF Alert' tab
\end{itemize} 

If there is a medical emergency during class, students may need to access a
first-aid kit or AED (Automated External Defibrillator). Here is the \href{http://www.ehs.ucf.edu/AEDlocations-UCF}{link} to learn where those are located.\\

To learn about how to manage an active-shooter situation on campus or elsewhere, consider viewing this \href{https://www.youtube.com/watch?v=NIKYajEx4pk&feature=youtu.be}{video}.\\

Students with special needs related to emergency situations should speak with me outside of class.
\subsubsection*{Extra Credit}
I do not provide ``extra credit'' that can only impact one individual. I may provide extra credit during class or for participating in an activity outside of class. I also provide additional ways to increase your grade in the course: dropped quizzes, homeworks, and the ability improve a bad midterm by performing better on the final.
\subsubsection*{Make-up Exams and Assignments}
Per university policy, students may only turn in make-up work (or and equivalent, alternate assignment) for \textbf{university-sponsored events, religious observances, or legal obligations (i.e. jury duty)}. In these instances, students are excused without penalty.\\

Students who know they will be absent due to a religious observance must notify me at the beginning of the semester so that make-up work can be arranged. For more information, please refer to the \href{https://regulations.ucf.edu/docs/notices/5.020ReligiousObservancesNEW_Oct09_000.pdf}{policy}.

\subsubsection*{Revisions to the Syllabus}
The contents of this syllabus may change as we progress through the semester. I will post an announcement declaring any major changes to the syllabus through Webcourses.
\subsubsection*{Student Academic Activity Policy}
As of Fall 2014, all faculty members are required to document the student's academic activity at the beginning of the course. How I am documenting it in this course is the first quiz. Please complete the quiz by the end of the first week, or you may delay/lose your financial aid. 
\subsubsection*{Student Accessibility Services}
The University of Central Florida is committed to providing access and inclusion for all persons with disabilities. Students with disabilities who need disability-related access in this course should contact the lecturer as soon as possible. Students should also connect with \href{https://sas.sdes.ucf.edu}{Student Accessibility Services} located at Ferrell Commons room 185, by \href{mailto:emailsas@ucf.edu}{email}, or phone 407-823-2371. Through Student Accessibility Services, a Course Accessibility Letter may be created and sent to professors, which informs faculty of potential access and accommodations that might be reasonable. Determining reasonable access and accommodations requires consideration of the course design, course learning objectives and the individual academic and course barriers experienced by the student.
\end{document}